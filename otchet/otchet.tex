\documentclass[a4paper,12pt]{article} % тип документа
\usepackage{indentfirst}
\usepackage{cmap}
\usepackage[T2A]{fontenc}			% кодировка
\usepackage[utf8]{inputenc}			% кодировка исходного текста
\usepackage[english,russian]{babel}	% локализация и переносы
\usepackage{amsmath,amsfonts,amssymb,amsthm,mathtools,array} 
\usepackage{wasysym}
\usepackage[labelsep=period]{caption}
\usepackage{graphicx}
\usepackage{pgfplots}
\usepackage{makeidx}
\usepackage{tikz}
\usepackage{pgfplots}
\pgfplotsset{compat=1.18}

\author{Кузнецов Игорь}
\title{Лабораторная работа по ТерМеху-1. Вариант 12}
\date{\today}

\newcolumntype{L}{>{$}l<{$}} % math-mode version of "l" column type
\newcommand{\ZE}{\bar{E}}
\newcommand{\BE}{\partial E}
\newcommand{\CE}{\complement E}
\newcommand{\IE}{\stackrel{\circ}{E}}
\newcommand{\Def}{\textbf{Определение }}
\newcommand{\Ter}{\textbf{Теорема }}
\newcommand{\Utv}{\textbf{Утверждение }}
\newcommand{\Prd}{\textbf{Предложение }}
\newcommand{\Dvo}{\textbf{Доказательство }}
\newcommand{\Imp}{\textbf{(!) }}
\newcommand{\Sld}{\textbf{Следствия: }}
\newcommand{\Svv}[1]{\textbf{Свойства #1:} }
\DeclareMathOperator{\Ree}{Re}
\DeclareMathOperator{\Imm}{Im}
\DeclareMathOperator{\res}{res}
\DeclareMathOperator{\cov}{cov\,}
\DeclareMathOperator{\kH}{\text{кН}}
\DeclareMathOperator{\m}{\text{м}}
\DeclareMathOperator{\kHm}{\kH\cdot\m}

\begin{document}
% \maketitle
\newcommand{\brv}[1]{{\left| #1 \right|}}
\newcommand{\brr}[1]{{\left( #1 \right)}}
\newcommand{\brs}[1]{{\left[ #1 \right]}}
\newcommand{\brc}[1]{{\left\{ #1 \right\}}}
\newcommand{\brn}[1]{{\left\lVert #1 \right\rVert}}
\newcommand{\bra}[1]{{\left\langle #1 \right\rangle}}
\newcommand{\brrl}[1]{{\left( #1 \right]}}
\newcommand{\brrr}[1]{{\left[ #1 \right)}}
\newcommand{\under}[2]{{\underset{#2}{\underbrace{#1}}}}
\newcommand{\strm}[1]{\underset{#1}{\rightarrow}}
\tableofcontents
\newpage

\section{Введение}

В последние годы нейронные сети получили широкое распространение, они широко используются для анализа и генерации изображений и видео, обработки естественных языков (перевод, чат-боты), медицинской диагностике, финансовых прогнозах и так далее.

Одни из перспективных направлений в этой области являются так называемые PINN -- Physics-Informed Neural Networks, физически-инфор\-мированные нейронные сети. Классические нейронные сети используют большую выборку реальных данных, однако в области биологии, химии и физике зачастую может просто не хватать нужного объёма данных для обучения. PINN способны обойти это ограничения, используя в обучении знания законов физики, описываемые дифференциальными уравнениями в частных производных.
\newpage

\section{Постановка задачи}

\subsection{Цель работы}

Рассмотреть несколько уже решённых физических задач, решить их с помощью PINN и сравнить полученные данные с изначальным решением, оценить целесообразность применения PINN к задаче.

\subsection{Описание задачи}

На вход нейросети подаются пространственно-временные координаты. На выходе хотим получить различные характеристики исследуемого физического процесса.
\newpage

\section{Теоретическое описание PINN}

В данное работе мы будем использовать простую сеть с прямой связью. Пусть система описывается неким системой дифференциальных уравнений 
\begin{equation}F_j(\lambda, u) = 0, X\in\Omega, t > 0\end{equation}
и набором граничных условий 
\begin{equation}u(t_0, x_0) = u_0\end{equation} 
где $x$ -- пространственные координаты, $\Omega$ -- некоторая область в $\mathbb{R}^n$, $t$ -- время, $u(t,x)$ -- искомая функция описывающая интересующие нас свойства системы (скорость, плотность, потенциал и т.п.), $\lambda$ -- вектор параметров. Для того что бы нейросеть могла обучаться на заданных уравнениях включим эти функции в функцию потерь в виде среднеквадратичной ошибки 
\begin{equation}
    MSE = MSE_F + MSE_0
\end{equation}
где
\begin{equation}
    MSE_F = \sum_j\frac{1}{N_F}\sum_{i=1}^{N_F} (F_j(u(t^i_f, x^i_f)))^2
\end{equation}
требует соблюдения дифуров, описывающих процесс, здесь $\brc{t^i_f, x^i_f}_{i=1}^{N_F}$ -- точки коллокации для $F_i$ и 
\begin{equation}
    MSE_0 = \frac{1}{N_0}\sum_{i=1}^{N_0} ((u(t^i_0, x^i_0)) - u^i_0)^2
\end{equation}
требует соблюдения граничных условий, $\brr{t^i_0, x^i_0, u^i_0}_{i=1}^{N_0}$ -- начальные и граничные условия $u(t,x)$.

\end{document}