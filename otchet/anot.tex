

\begin{center}
    
    \section*{РЕФЕРАТ}
    \addcontentsline{toc}{section}{РЕФЕРАТ}
\end{center}

Кузнецов Игорь александрович <<Возможность использования искусственных нейронных сетей для решения задач математической физики>>:  работа содержит: страниц \total{page}, иллюстраций \total{figure}, использованных источников \total{citnum}

\noindent Ключевые слова: PINN, дифференциальные уравнения, электрокинетика, нейронные сети, python

% Целью работы является исследование возможности применения PINN (Physics-Informed Neural Networks) в решении задач 
PINN (Physics-Informed Neural Networks) - это метод, который сочетает в себе преимущества нейронных сетей и физических моделей для решения задач научного моделирования. В данном дипломном проекте исследуется применение метода PINN для решения задачи обратной задачи электрокинетики. В работе проводится анализ эффективности метода PINN в сравнении с классическими методами решения обратных задач. Также исследуется влияние различных параметров на точность рения задачи. Результаты исследования показывают, что метод PINN может быть эффективным инструментом для решения задач научного моделирования, особенно в случаях, когда классические методы неэффективны или недостаточно точны.
Были использованы следующие технологии:
\begin{itemize}
    \item язык программирования Python;
    \item библиотека для работы с глубокими нейронными сетями Tensorflow;
    \item библиотека ESPResSo, для проведения традиционных симуляций;
\end{itemize}
\newpage