\section*{Обозначения и сокращения}

В настоящей работе применяют следующие обозначения и сокращения

\begin{description}
    \item[PINN] -- Physics-Informed neural network, физически информированная нейронная сеть
    \item[$c$] -- Концентрация ионных частиц
    \item[$j$] -- Поток плотности
    \item[$\vec{v}$] -- Адвективная скорость жидкости
    \item[$e$] -- Заряд электрона
    \item[$z$] -- Валентность частиц
    \item[$\Phi$] -- Электростатический потенциал
    \item[$\xi$] -- Подвижность частиц
    \item[$D$] -- Коэффициент диффузии частиц
    \item[$l_B$] -- Длина Бьеррума, $l_B = \frac{e^2}{4\pi\varepsilon k_B T}$
    \item[$k_B$] -- Постоянная Больцмана
    \item[$T$] -- Температура
    \item[$\rho$] -- Плотность жидкости
    \item[$p_H$] -- Гидродинамическое давление
\end{description}

% \noindent\begin{tabular}{llm{15cm}}
%     PINN & -- & Physics-Informed neural network, физически информированная нейронная сеть
% \end{tabular}

\newpage

% \\nomenclature \{(\$[a-z\\\{\}_]*\$)\}\{(.*)\}
% \\item[$1] -- $2
% \nomenclature {$c$}{Концентрация ионных частиц}
% \nomenclature {$j$}{Поток плотности}
% \nomenclature {$\vec{v}$}{Адвективная скорость жидкости}
% \nomenclature {$e$}{Заряд электрона}
% \nomenclature {$z$}{Валентность частиц}
% \nomenclature {$\Phi$}{Электростатический потенциал}
% \nomenclature {$\xi$}{Подвижность частиц}
% \nomenclature {$D$}{Коэффициент диффузии частиц}
% \nomenclature {$l_B$}{Длина Бьеррума, $l_B = \frac{e^2}{4\pi\varepsilon k_B T}$}
% \nomenclature {$k_B$}{Постоянная Больцмана}
% \nomenclature {$T$}{Температура}
% \nomenclature {$\rho$}{Плотность жидкости}
% \nomenclature {$p_H$}{Гидродинамическое давление}

% \printnomenclature
